% Options for packages loaded elsewhere
\PassOptionsToPackage{unicode}{hyperref}
\PassOptionsToPackage{hyphens}{url}
\PassOptionsToPackage{dvipsnames,svgnames,x11names}{xcolor}
%
\documentclass[
  letterpaper,
  DIV=11,
  numbers=noendperiod]{scrartcl}

\usepackage{amsmath,amssymb}
\usepackage{iftex}
\ifPDFTeX
  \usepackage[T1]{fontenc}
  \usepackage[utf8]{inputenc}
  \usepackage{textcomp} % provide euro and other symbols
\else % if luatex or xetex
  \usepackage{unicode-math}
  \defaultfontfeatures{Scale=MatchLowercase}
  \defaultfontfeatures[\rmfamily]{Ligatures=TeX,Scale=1}
\fi
\usepackage{lmodern}
\ifPDFTeX\else  
    % xetex/luatex font selection
\fi
% Use upquote if available, for straight quotes in verbatim environments
\IfFileExists{upquote.sty}{\usepackage{upquote}}{}
\IfFileExists{microtype.sty}{% use microtype if available
  \usepackage[]{microtype}
  \UseMicrotypeSet[protrusion]{basicmath} % disable protrusion for tt fonts
}{}
\makeatletter
\@ifundefined{KOMAClassName}{% if non-KOMA class
  \IfFileExists{parskip.sty}{%
    \usepackage{parskip}
  }{% else
    \setlength{\parindent}{0pt}
    \setlength{\parskip}{6pt plus 2pt minus 1pt}}
}{% if KOMA class
  \KOMAoptions{parskip=half}}
\makeatother
\usepackage{xcolor}
\usepackage{soul}
\setlength{\emergencystretch}{3em} % prevent overfull lines
\setcounter{secnumdepth}{-\maxdimen} % remove section numbering
% Make \paragraph and \subparagraph free-standing
\ifx\paragraph\undefined\else
  \let\oldparagraph\paragraph
  \renewcommand{\paragraph}[1]{\oldparagraph{#1}\mbox{}}
\fi
\ifx\subparagraph\undefined\else
  \let\oldsubparagraph\subparagraph
  \renewcommand{\subparagraph}[1]{\oldsubparagraph{#1}\mbox{}}
\fi


\providecommand{\tightlist}{%
  \setlength{\itemsep}{0pt}\setlength{\parskip}{0pt}}\usepackage{longtable,booktabs,array}
\usepackage{calc} % for calculating minipage widths
% Correct order of tables after \paragraph or \subparagraph
\usepackage{etoolbox}
\makeatletter
\patchcmd\longtable{\par}{\if@noskipsec\mbox{}\fi\par}{}{}
\makeatother
% Allow footnotes in longtable head/foot
\IfFileExists{footnotehyper.sty}{\usepackage{footnotehyper}}{\usepackage{footnote}}
\makesavenoteenv{longtable}
\usepackage{graphicx}
\makeatletter
\def\maxwidth{\ifdim\Gin@nat@width>\linewidth\linewidth\else\Gin@nat@width\fi}
\def\maxheight{\ifdim\Gin@nat@height>\textheight\textheight\else\Gin@nat@height\fi}
\makeatother
% Scale images if necessary, so that they will not overflow the page
% margins by default, and it is still possible to overwrite the defaults
% using explicit options in \includegraphics[width, height, ...]{}
\setkeys{Gin}{width=\maxwidth,height=\maxheight,keepaspectratio}
% Set default figure placement to htbp
\makeatletter
\def\fps@figure{htbp}
\makeatother

\KOMAoption{captions}{tableheading}
\makeatletter
\makeatother
\makeatletter
\makeatother
\makeatletter
\@ifpackageloaded{caption}{}{\usepackage{caption}}
\AtBeginDocument{%
\ifdefined\contentsname
  \renewcommand*\contentsname{Table of contents}
\else
  \newcommand\contentsname{Table of contents}
\fi
\ifdefined\listfigurename
  \renewcommand*\listfigurename{List of Figures}
\else
  \newcommand\listfigurename{List of Figures}
\fi
\ifdefined\listtablename
  \renewcommand*\listtablename{List of Tables}
\else
  \newcommand\listtablename{List of Tables}
\fi
\ifdefined\figurename
  \renewcommand*\figurename{Figure}
\else
  \newcommand\figurename{Figure}
\fi
\ifdefined\tablename
  \renewcommand*\tablename{Table}
\else
  \newcommand\tablename{Table}
\fi
}
\@ifpackageloaded{float}{}{\usepackage{float}}
\floatstyle{ruled}
\@ifundefined{c@chapter}{\newfloat{codelisting}{h}{lop}}{\newfloat{codelisting}{h}{lop}[chapter]}
\floatname{codelisting}{Listing}
\newcommand*\listoflistings{\listof{codelisting}{List of Listings}}
\makeatother
\makeatletter
\@ifpackageloaded{caption}{}{\usepackage{caption}}
\@ifpackageloaded{subcaption}{}{\usepackage{subcaption}}
\makeatother
\makeatletter
\@ifpackageloaded{tcolorbox}{}{\usepackage[skins,breakable]{tcolorbox}}
\makeatother
\makeatletter
\@ifundefined{shadecolor}{\definecolor{shadecolor}{rgb}{.97, .97, .97}}
\makeatother
\makeatletter
\makeatother
\makeatletter
\makeatother
\makeatletter
\@ifpackageloaded{fontawesome5}{}{\usepackage{fontawesome5}}
\makeatother
\ifLuaTeX
  \usepackage{selnolig}  % disable illegal ligatures
\fi
\IfFileExists{bookmark.sty}{\usepackage{bookmark}}{\usepackage{hyperref}}
\IfFileExists{xurl.sty}{\usepackage{xurl}}{} % add URL line breaks if available
\urlstyle{same} % disable monospaced font for URLs
\hypersetup{
  pdftitle={Syllabus},
  colorlinks=true,
  linkcolor={blue},
  filecolor={Maroon},
  citecolor={Blue},
  urlcolor={Blue},
  pdfcreator={LaTeX via pandoc}}

\title{Syllabus}
\author{}
\date{}

\begin{document}
\maketitle
\ifdefined\Shaded\renewenvironment{Shaded}{\begin{tcolorbox}[enhanced, boxrule=0pt, interior hidden, breakable, borderline west={3pt}{0pt}{shadecolor}, sharp corners, frame hidden]}{\end{tcolorbox}}\fi

\hypertarget{instructor}{%
\subsubsection{Instructor}\label{instructor}}

\begin{itemize}
\tightlist
\item
  \faIcon{user} ~ \href{https://justinsmithecon.github.io}{Prof.~Justin
  Smith}
\item
  \faIcon{university} ~ Lazaridis Hall 3091
\item
  \faIcon{envelope} ~ jusmith@wlu.ca
\end{itemize}

\hypertarget{course-details}{%
\subsubsection{Course details}\label{course-details}}

\begin{itemize}
\tightlist
\item
  \faIcon{calendar} ~ Mon/Wed
\item
  \faIcon{calendar-alt} ~ Sept-Dec 2023
\item
  \faIcon{clock} ~ 1:00 - 2:20 PM
\item
  \faIcon{location-dot} ~ P118
\end{itemize}

\hypertarget{office-hours}{%
\subsubsection{Office Hours}\label{office-hours}}

\begin{itemize}
\tightlist
\item
  \faIcon{clock} ~ Thur 2:30 PM - 4:30 PM
\item
  \faIcon{calendar-check} ~
  \href{https://outlook.office365.com/owa/calendar/BookaTimewithDrJustinSmith@wlu.ca/bookings/}{Schedule
  an appointment}
\end{itemize}

\hypertarget{course-description}{%
\subsection{Course Description}\label{course-description}}

This first course in econometrics at the graduate level will build on
the knowledge you gained in your undergraduate econometrics classes.
Some of the topics we cover will be advanced versions of things you
already know, while others will be completely new. The goal is to build
a foundation in economic statistics for those who want pursue a career
in data analysis and also those who will continue to study economics at
the PhD level. While the course will cover both theoretical and
empirical aspects of econometrics, it will have an applied focus. What
this means operationally is that assignments and exams are geared
towards applying your knowledge to real world economic situations,
estimating models using data, and discussing model intuition.

\hypertarget{lectures}{%
\subsection{Lectures}\label{lectures}}

The scheduling details of the course are as follows

\begin{longtable}[]{@{}ccc@{}}
\toprule\noalign{}
Section & Time & Location \\
\midrule\noalign{}
\endhead
\bottomrule\noalign{}
\endlastfoot
A & MW 1:00-2:20pm & P118 \\
\end{longtable}

\hypertarget{course-material}{%
\subsection{Course Material}\label{course-material}}

\hypertarget{textbook}{%
\subsubsection{Textbook}\label{textbook}}

\textbf{There is no required textbook} for this course, but many
supplemental readings will be drawn from the following list:

\begin{itemize}
\item
  (AP1) Angrist, Joshua D. and Jörn-Steffen Pischke, \emph{Mostly
  Harmless Econometrics: An Empiricist's Companion}. Princeton:
  Princeton University Press, 2009.
\item
  (AP2) Angrist, Joshua D. and Jörn-Steffen Pischke, \emph{Mastering
  Metrics: The Path From Cause to Effect}. Princeton: Princeton
  University Press, 2015.
\item
  (CT) Cameron, A. Colin and Pravin K. Trivedi,
  \emph{Microeconometrics}. New York: Cambridge University Press, 2005.
\item
  (C) Cunningham, Scott, Causal Inference: The Mixtape. Available at:
  \url{https://mixtape.scunning.com}, 2021.
\item
  (HA) Hansen, Bruce E., \emph{Econometrics}. Princeton: Princeton
  University Press, 2022.
\item
  (HK) Huntington-Klein, Nick, The Effect. Available at
  \url{https://theeffectbook.net}, 2021.
\item
  (K) Kennedy, Peter, \emph{A Guide to Econometrics} 6E. Malden:
  Blackwell Publishing, 2008.
\item
  (SW) Stock, James H., and Mark M. Watson (2015). \emph{Introduction to
  Econometrics}, 4\textsuperscript{th} Edition. Pearson Education.
\item
  (W1) Wooldridge, Jeffrey M., \emph{Econometric Analysis of
  Cross-Section and Panel Data}, Second Edition. Cambridge: MIT Press,
  2010.
\item
  (W2) Wooldridge, Jeffrey M., Introductory \emph{Econometrics: A Modern
  Approach Analysis of Cross-Section and Panel Data}, Seventh Edition.
  Mason: South-Western, Cengage, 2019.
\end{itemize}

\hypertarget{software}{%
\subsubsection{Software}\label{software}}

All assignments require you to manipulate data using the statistical
software R alongside a user interface called R Studio. The assignments
are also required to be written in Quarto, which is intergrated with R
Studio. All of these programs are free, and available on all computer
platforms. You can use R without R Studio, but I would only recommend
this is you have prior experience with the program or have a computer
programming background. I will give instructions on how to access these
materials in class.

\hypertarget{evaluation}{%
\subsection{Evaluation}\label{evaluation}}

You will be evaluated on three equally weighted assignments, one
midterm, and one final exam. The weights and due dates for each
assessment are as follows:

\begin{longtable}[]{@{}lll@{}}
\toprule\noalign{}
Assessment & Due Date & Weight \\
\midrule\noalign{}
\endhead
\bottomrule\noalign{}
\endlastfoot
Assignment 1 & Friday, October 6, 2023 at 9:00pm & 16.67\% \\
Assignment 2 & Friday, November 3, 2023 at 9:00pm & 16.67\% \\
Midterm & Wednesday, November 8, 2023, in class & 20\% \\
Assignment 3 & Friday December 1, 2023 at 12:00pm & 16.67\% \\
Final Exam & TBA & 30\% \\
\end{longtable}

Assignments will ask you to manipulate and interpret data using the
statistical software R. Instructions will be posted to MLS at least one
week prior to the due date.

Both the midterm and final exam will be in person. The midterm will take
place in the classroom where the lectures take place at the scheduled
lecture time. The final exam schedule is posted roughly half way through
the term.

\hypertarget{topics}{%
\subsection{Topics}\label{topics}}

Below is a list of tentative topics covered in the course. I may add or
remove items depending on how quickly the course proceeds. A reading
list for each topic is available at the end of this syllabus.

\begin{enumerate}
\def\labelenumi{\arabic{enumi}.}
\item
  Introduction to R
\item
  Review of Matrix Algebra for Econometrics
\item
  Conditional Expecations, Linear Regression, and OLS
\item
  Causal Inference
\item
  Panel Data Methods
\item
  Qualitative Dependent Variable Models
\item
  Additional Topics
\end{enumerate}

\hypertarget{missed-midterms}{%
\subsection{Missed Midterms}\label{missed-midterms}}

Students who miss a midterm will be given reasonable accommodation for
the following reasons:

\begin{enumerate}
\def\labelenumi{\arabic{enumi})}
\item
  \ul{Religious conflict}: If you have a religious commitment that
  interferes with the midterm exam, university policy is that
  \textbf{you must alert me within the first two weeks of the start of
  the term} and fill out the
  \href{https://web.wlu.ca/accommodations/}{Student Request for
  Accommodation for Religious Observances form}. If those conditions are
  met, we will work together to provide a reasonable accommodation.
\item
  \ul{Course conflict}: If your midterm overlaps with another scheduled
  course, please inform me as soon as you know about the conflict and we
  will work out a reasonable accommodation.
\item
  \ul{Varsity Sports}: If a varsity sporting event interferes with the
  midterm, you must contact your coach, who will arrange for an
  alternative time to write the test. Note that this policy applies only
  to \emph{varsity} sports; students who have non-varsity sports
  conflicts are not eligible for a deferred midterm
\end{enumerate}

Students who miss a midterm and have an acceptable medical or
compassionate reason will have the weight of the midterm transferred to
the final exam. In the case of medical reasons, I require that students
complete the \href{https://web.wlu.ca/illness/}{Absence for Medical
Reasons Self-Declaration Form}.

In all other circumstances, students who miss a midterm will receive a
grade of zero on the test.

\hypertarget{late-assignments}{%
\subsection{Late Assignments}\label{late-assignments}}

Late assignments and quizzes for religious conflicts, course conflicts,
or medical/compassionate reasons as defined above will be given
reasonable accommodation. In all other circumstances, students who
submit late assignments or quizzes will receive a grade of zero.

\hypertarget{deferred-final-examinations}{%
\subsection{Deferred Final
Examinations}\label{deferred-final-examinations}}

Students who miss a final examination can submit a petition to the
Faculty of Graduate and Postdoctoral Studies for a deferred exam.
Students who miss the exam due to illness must submit the petition no
later than 5 days after the missed exam with original supporting
documentation. Students requesting a deferred exam for reasons other
than personal illness or bereavement must submit a petition at the time
of the exam schedule posting to allow the committee to reach a decision
before the scheduled date.

Students are strongly urged not to make any commitments (e.g., vacation)
during the examination period. Students are required to be available for
examinations during the examination periods of all terms in which they
register. Refer to the Handbook on Undergraduate Course Management for
more information.

\hypertarget{academic-integrity}{%
\subsection{Academic Integrity}\label{academic-integrity}}

You are reminded that the University will levy sanctions on students who
are found to have committed, or have attempted to commit, acts of
academic or research misconduct. You are expected to know what
constitutes an academic offense, to avoid committing such offenses, and
to take responsibility for your academic actions. For information on
categories of offenses and types of penalty, please consult the relevant
section of the Undergraduate Academic Calendar. If you need
clarification of aspects of University policy on Academic and Research
Misconduct, please consult your instructor.

Wilfrid Laurier University uses software that can check for plagiarism.
Students may be required to submit their written work in electronic form
and have it checked for plagiarism.

\hypertarget{accessible-learning}{%
\subsection{Accessible Learning}\label{accessible-learning}}

Students with disabilities or special needs are advised to contact
Laurier's Accessible Learning Centre for information regarding its
services and resources. Students should review the Calendar for
information regarding all services available on campus.

\hypertarget{student-privacy}{%
\subsection{Student Privacy}\label{student-privacy}}

Wilfrid Laurier University uses a range of technologies to facilitate
in-person and remote instruction. Zoom is currently used for remote
course delivery, including lectures, seminars, and group office hours,
which may be recorded, stored and shared through MyLearningSpace for
access by students in the course. For these course activities,students
are permitted to turn off their cameras or use an alternative name to
maintain their privacy after they have confirmed this with their
instructor. Student personal information is collected and used in the
course in accordance with University policies and the
\href{https://www.wlu.ca/about/public-accountability/privacy/notice-of-collection.html}{Notice
of Collection, Use or Disclosure of Personal Information}.

Some synchronous (live) class sessions will be delivered in this course
through a video conferencing platform supported by the university
{[}Zoom, Teams, Virtual Classroom{]}. Steps have been taken to protect
the security of the information shared. For more information about Zoom
and Office365 (including Teams), please visit ICT's Tech Support and
Services page. Class sessions will be recorded with the video and audio
(and in some cases transcription) made available to students in the
course in MyLearningSpace for the duration of the term. The recordings
may capture your name, image or voice through the video and audio
recordings. By attending in these live classes, you are consenting to
the collection of this information for the purposes of administering the
class and associated course work. If you are concerned about the use or
collection of your name and other personal information in the class,
please contact the course instructor to identify possible alternatives.
To learn more about how your personal information is collected, used and
disclosed by the University, please see Laurier's Notice of Collection,
Use and Disclosure of Personal Information.

\hypertarget{intellectual-property}{%
\subsection{Intellectual Property}\label{intellectual-property}}

The educational materials developed for this course, including, but not
limited to, lecture notes and slides, handout materials, examinations
and assignments, and any materials posted to MyLearningSpace, are the
intellectual property of the course instructor. These materials have
been developed for student use only and they are not intended for wider
dissemination and/or communication outside of a given course. Posting or
providing unauthorized audio, video, or textual material of lecture
content to third-party websites violates an instructor's intellectual
property rights, and the Canadian Copyright Act. Recording lectures in
any way is prohibited in this course unless specific permission has been
granted by the instructor. Failure to follow these instructions may be
in contravention of the university's Student Non-Academic Code of
Conduct and/or Code of Academic Conduct, and will result in appropriate
penalties. Participation in this course constitutes an agreement by all
parties to abide by the relevant University Policies, and to respect the
intellectual property of others during and after their association with
Wilfrid Laurier University.

\hypertarget{foot-patrol-the-wellness-centre-and-the-student-food-bank}{%
\subsection{Foot Patrol, the Wellness Centre, and the Student Food
Bank}\label{foot-patrol-the-wellness-centre-and-the-student-food-bank}}

The University approved the inclusion of information about select
wellness and safety services and supports on campus in the course
information provided to students. (Approved by Senate November 28,
2011.) Specific language (by campus) is provided below.

\begin{itemize}
\item
  Multi-campus Resource:

  \begin{itemize}
  \tightlist
  \item
    Good2Talk is a postsecondary school helpline that provides free,
    professional and confidential counselling support for students in
    Ontario. Call 1-866-925-5454 or through 2-1-1. Available 24-7.
  \end{itemize}
\item
  Kitchener/Waterloo Resources:

  \begin{itemize}
  \item
    \href{http://yourstudentsunion.ca/service/food-bank/}{Waterloo
    Student Food Bank}: All students are eligible to use this service to
    ensure they're eating healthy when overwhelmed, stressed or
    financially strained. Anonymously request a package online 24-7. All
    dietary restrictions accommodated.
  \item
    \href{http://yourstudentsunion.ca/service/foot-patrol/}{Waterloo
    Foot Patrol}: 519.886.FOOT (3668). A volunteer operated safe-walk
    program, available Fall and Winter daily from 6:30 pm to 3 am. Teams
    of two are assigned to escort students to and from campus by foot or
    by van.
  \item
    \href{https://students.wlu.ca/wellness-and-recreation/health-and-wellness/index.html}{Waterloo
    Student Wellness Centre}:519-884-0710, x3146. The Centre supports
    the physical, emotional, and mental health needs of students.
    Located on the 2\textsuperscript{nd} floor of the Student Services
    Building, booked and same-day appointments are available Mondays and
    Wednesdays from 8:30 am to 7:30 pm, and Tuesdays, Thursdays and
    Fridays from 8:30 am to 4:15 pm. Contact the Centre at x3146,
    \href{mailto:wellness@wlu.ca}{\nolinkurl{wellness@wlu.ca}} or
    @LaurierWellness. After hours crisis support available 24/7. Call
    1-844-437-3247 (HERE247).
  \end{itemize}
\item
  Brantford Resources:

  \begin{itemize}
  \item
    \href{http://yourstudentsunion.ca/service/food-bank/}{Brantford
    Student Food Bank}: All students are eligible to use this service to
    ensure they're eating healthy when overwhelmed, stressed or
    financially strained. Anonymously request a package online 24-7. All
    dietary restrictions accommodated.
  \item
    \href{https://students.wlu.ca/wellness-and-recreation/safety/foot-patrol.html}{Brantford
    Foot Patrol}:519-751-PTRL (7875). A volunteer operated safe-walk
    program, available Fall and Winter, Monday through Thursday from
    6:30 pm to 1 am; Friday through Sunday 6:30 pm to 11 pm. Teams of
    two are assigned to escort students to and from campus by foot or by
    van.
  \item
    \href{https://students.wlu.ca/wellness-and-recreation/health-and-wellness/index.html}{Brantford
    Wellness Centre}:519-756-8228, x5803. Students have access to
    support for all their physical, emotional, and mental health needs
    at the Wellness Centre. Location: Student Centre, 2nd floor. Hours:
    8:30 am to 4:15 pm Monday through Friday. After hours crisis support
    available 24/7. Call 1-884-437-3247 (HERE247).
  \end{itemize}
\end{itemize}



\end{document}
